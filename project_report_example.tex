\documentclass[twocolumn]{article}

\usepackage{graphicx}
\usepackage{amsmath}
\usepackage{amsthm}
\usepackage{amssymb}
\usepackage{url}
\usepackage{multirow}
\usepackage{times}
\usepackage{fullpage}

\newcommand{\comment}[1]{}

\title{CS315: Group xx \\
An Example Project Report}
\author{
\begin{tabular}{ccc}
	Student Name 1 & Student Name 2 & Student Name 3 \\
	Student Id 1 & Student Id 2 & Student Id 3 \\
	Roll Number 1 & Roll Number 2 & Roll Number 3 \\
	\url{email1@iitk.ac.in} & \url{email2@iitk.ac.in} & \url{email3@iitk.ac.in} \\
	Dept. of CSE & Dept. of XYZ & Dept. of ABC \\
	\multicolumn{3}{c}{Indian Institute of Technology, Kanpur}
\end{tabular}
}
\date{Final report \\	% replace by ``initial'' or ``final'' as appropriate
11th April, 2014}	% replace by actual date of submission or \today

\begin{document}

\maketitle

\begin{abstract}
	%
	Abstract of the project.
	%
\end{abstract}

\section{Introduction and Problem Statement}

State the problem as clearly and as formally as possible.  Explain the
notations, etc.  Explain the objectives, and all the inputs.  If possible,
motivate why this is an interesting problem.

\subsection{Related Material}

Fill in all relevant work, datasets, websites, etc.

\comment{

Can also comment out paragraphs, etc.

}

\section{Algorithm or Approach}

Details of the method.
Put in a pseudo-code, etc. if applicable.
Explain with figures.

\comment{

Use the following format for figures:

\begin{figure}[t]
	\centering
	\includegraphics[width=0.95\columnwidth]{figure_file}
	\caption{This figure explains this.}
	\label{fig:block}
\end{figure}

And refer as Figure \ref{fig:block}.

}

\section{Results}

Details of results, in tabular and/or graphical formats.

More importantly, analyze the results.

\comment{

\begin{table}[t]
	\centering
	\begin{tabular}{|c||cc|}
		\hline
		Header 1 & Desc 1 & Desc 2 \\
		\hline
		\hline
		Row 1 & Data 1-1 & Data 1-2 \\
		Row 2 & Data 2-1 & Data 2-2 \\
		\hline
	\end{tabular}
	\caption{Table of results.}
	\label{tab:results}
\end{table}

And refer as Table \ref{tab:results}.

}

\section{Conclusions}

Clearly state the conclusions.
Outline the future work.

\section*{References}

Directly type in bib entries.

Better is to use bibtex.

\end{document}

